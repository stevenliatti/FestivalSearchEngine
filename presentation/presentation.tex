\documentclass{beamer}
\usepackage[utf8]{inputenc}
\usepackage[francais]{babel}
\usepackage[T1]{fontenc}
\usepackage[export]{adjustbox}
\newcommand\Fontvi{\fontsize{8}{7.2}\selectfont}

\usepackage{minted}
\usemintedstyle{colorful}
\usepackage{hyperref} 
\hypersetup{
	colorlinks,
	citecolor=black,
	filecolor=black,
	linkcolor=black,
	urlcolor=blue
}

\usetheme{Warsaw}

\addtobeamertemplate{navigation symbols}{}{%
    \usebeamerfont{footline}%
    \usebeamercolor[fg]{footline}%
    \hspace{1em}%
    \insertframenumber/\inserttotalframenumber
}

\begin{document}
\logo{%
	\makebox[0.95\paperwidth]{%
		\includegraphics[width=2.5cm,keepaspectratio]{images/hepia.jpg}%
		\hfill%
		\includegraphics[width=2.5cm,keepaspectratio]{images/hesso.jpg}%
	}%
}

\title{Festival Search Engine}
\author{Steven Liatti et Vincent Tournier}
\institute{Cours de systèmes distribués - Prof. Nabil Abdennadher - Hepia ITI 3\up{ème} année}
\date{13 décembre 2017}

\begin{frame}
\titlepage
\end{frame}

\begin{frame}
	\setcounter{tocdepth}{3}
	\tableofcontents
\end{frame}

\section{Introduction}
\subsection{API utilisées}
\begin{frame}
	\frametitle{\secname}
	\framesubtitle{\subsecname}
	\begin{itemize}
		\item Spotify : recherche d'artistes et top tracks (route \mintinline{text}{events}, \mintinline{text}{infos} et \mintinline{text}{tracks})
		\item Eventful : liste des événements (lieux, dates et artistes) (route \mintinline{text}{events})
		\item Wikipédia : principale source d'informations sur un artiste (route \mintinline{text}{infos})
        \item MusicBrainz : informations complémentaires sur les artistes (route \mintinline{text}{infos})
		\item BandsInTown : informations complémentaires sur les artistes (route \mintinline{text}{events} et \mintinline{text}{infos})
	\end{itemize}
\end{frame}

\section{Serveur}
\subsection{Généralités}
\subsubsection{Promesses}
\begin{frame}
	\frametitle{\subsecname}
	\framesubtitle{\subsubsecname}    
	\begin{figure}
		\begin{center}
			\includegraphics[width=1.0\textwidth]{images/hepia.jpg}
		\end{center}
		\caption{hepia}
		% \label{conso_ordis}
	\end{figure}
\end{frame}


\subsection{Route \mintinline{text}{events}}
\begin{frame}
	\frametitle{\subsecname}
	\framesubtitle{\subsubsecname}
	\begin{columns}[T]
		\begin{column}{.25\textwidth}
			\begin{figure}
				\includegraphics[width=0.6\textwidth]{images/hesso.jpg}
				\caption{hesso}
			\end{figure}
		\end{column}
		\begin{column}{.75\textwidth}
			\begin{itemize}
				\item 2 colonnes
			\end{itemize}
		\end{column}
	\end{columns}
\end{frame}

\subsection{Route \mintinline{text}{infos}}
\subsection{Route \mintinline{text}{tracks}}
\subsection{Déployé sur AWS}

\section{Client}

\section{Conclusion}
\begin{frame}
	\frametitle{\secname}
	\begin{itemize}
		\item 
	\end{itemize}
	\Large\textbf{texte important}
\end{frame}

\end{document}
