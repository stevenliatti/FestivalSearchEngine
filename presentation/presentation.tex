\documentclass[10pt]{beamer}
\usepackage[utf8]{inputenc}
\usepackage[francais]{babel}
\usepackage[T1]{fontenc}
\usepackage[export]{adjustbox}
\newcommand\Fontvi{\fontsize{8}{7.2}\selectfont}

\usepackage{minted}
\usemintedstyle{colorful}
\usepackage{hyperref}
\hypersetup{
	colorlinks,
	citecolor=black,
	filecolor=black,
	linkcolor=black,
	urlcolor=blue
}

\usetheme{Frankfurt}
\usecolortheme{beaver}

\addtobeamertemplate{navigation symbols}{}{%
    \usebeamerfont{footline}%
    \usebeamercolor[fg]{footline}%
    \hspace{1em}%
    \insertframenumber/\inserttotalframenumber
}

\begin{document}
\logo{%
	\makebox[0.95\paperwidth]{%
		\includegraphics[width=2.5cm,keepaspectratio]{images/hepia.jpg}%
		\hfill%
		\includegraphics[width=2.5cm,keepaspectratio]{images/hesso.jpg}%
	}%
}

\title{Festival Search Engine}
\author{Steven Liatti et Vincent Tournier}
\institute{Cours de systèmes distribués - Prof. Nabil Abdennadher - Hepia ITI 3\up{ème} année}
\date{13 décembre 2017}

\begin{frame}
\titlepage
\end{frame}

\begin{frame}
	\setcounter{tocdepth}{2}
    \frametitle{Plan}
    \begin{columns}[t]
        \begin{column}{.5\textwidth}
            \tableofcontents[sections={1-2}]
        \end{column}
        \begin{column}{.5\textwidth}
            \tableofcontents[sections={3-4}]
        \end{column}
    \end{columns}
\end{frame}

\section{Introduction}
\subsection{Buts du projet}
\begin{frame}
	\frametitle{\secname}
	\framesubtitle{\subsecname}
	Créer un moteur de recherche d'événements musicaux, permettant à l'utilisateur de :
	\begin{itemize}
		\item Afficher des événements sur une carte interactive
		\item Afficher des informations d'un événement en particulier
		\item Afficher des informations à propos des artistes
		\item Jouer (en arrière plan) un extrait d'un son d'un artiste de l'événement
	\end{itemize}
\end{frame}

\subsection{APIs utilisées}
\begin{frame}
	\frametitle{\secname}
	\framesubtitle{\subsecname}
	\begin{itemize}
		\item Spotify : recherche d'artistes et top tracks (route \mintinline{text}{events}, \mintinline{text}{infos} et \mintinline{text}{tracks})
		\item Eventful : liste des événements (lieux, dates et artistes) (route \mintinline{text}{events})
		\item Wikipédia : principale source d'informations sur un artiste (route \mintinline{text}{infos})
        \item MusicBrainz : informations complémentaires sur les artistes (route \mintinline{text}{infos})
		\item BandsInTown : informations complémentaires sur les artistes (route \mintinline{text}{events} et \mintinline{text}{infos})
		\item Google Maps : pour la carte interactive côté client
	\end{itemize}
\end{frame}

\section{Serveur}
\subsection{Route \mintinline{text}{events}}
\begin{frame}
	\frametitle{\secname}
	\framesubtitle{\subsecname}
	\begin{figure}
		\begin{center}
			\includegraphics[width=0.8\textwidth]{images/events.png}
		\end{center}
		\caption{Route events}
	\end{figure}
\end{frame}

\subsection{Route \mintinline{text}{infos}}
\begin{frame}
	\frametitle{\secname}
	\framesubtitle{\subsecname}
	\begin{figure}
		\begin{center}
			\includegraphics[width=0.8\textwidth]{images/infos.png}
		\end{center}
		\caption{Route infos}
	\end{figure}
\end{frame}

\subsection{Route \mintinline{text}{tracks}}
\begin{frame}
	\frametitle{\secname}
	\framesubtitle{\subsecname}
	\begin{figure}
		\begin{center}
			\includegraphics[width=0.8\textwidth]{images/tracks.png}
		\end{center}
		\caption{Route tracks}
	\end{figure}
\end{frame}

\subsection{MongoDB}
\begin{frame}
	\frametitle{\secname}
	\framesubtitle{\subsecname}
	Utilisation de MongoDB comme base de données :
	\begin{itemize}
		\item Simplicité/compatibilité avec Javascript et Node.js (JSON)
		\item Données "temporaires" avec la feature MongoDB Time To Live
		\item Découverte d'un SGBD alternatif (NoSQL)
	\end{itemize}
\end{frame}

\section{Client}
\subsection{APIs \& technologies}
\begin{frame}
	\frametitle{\secname}
	\framesubtitle{\subsecname}
	\begin{itemize}
		\item Javascript \& JQuery
		\item Google Maps (Map, marker, infobubble, geocoder)
		\item Overlapping Marker Spiderfier
		\item Maps marker clusterer
	\end{itemize}
\end{frame}

\subsection{En gros}
\begin{frame}
	\frametitle{\secname}
	\framesubtitle{\subsecname}
	\begin{itemize}
		\item Un formulaire
		\item Pleins de <div>
		\item Deux arguments en URL
		\item Et du script
	\end{itemize}
\end{frame}

\section{Conclusion}
\begin{frame}
	\frametitle{\secname}
	Steven :
	\begin{itemize}
		\item Approfondissement de Node.js et découverte des Promise Javascript et de MongoDB
		\item Découverte de bonnes (Spotify, Wikipédia) et "moins bonnes" (MusicBrainz, Eventful) API
	\end{itemize}
	Vincent :
	\begin{itemize}
		\item
	\end{itemize}
	\Large\textbf{Beaucoup de plaisir à travailler sur ce projet}
\end{frame}

\end{document}
